

\newif\iftr     % Full arXiv technical report
\newif\ifconf   % Size-constrainted Submission to a conf or journal
\newif\ifnonb   % Non blind submission

%%%%%%%%%%%%%
% Various
%%%%%%%%%%%%%
\usepackage[utf8]{inputenc}
\newcommand{\code}[1]{\texttt{#1}}
% binary units
\usepackage[binary-units=true]{siunitx}
\usepackage{xspace}
\usepackage[svgnames]{xcolor}

%%%%%%%%%%%%%
% Notes
%%%%%%%%%%%%%
\usepackage{collab}
% Define author:
%\collabAuthor{cmd_name}{color}{display_name}

%%%%%%%%%%%%%
% Bibliography
%%%%%%%%%%%%%
% citep
%\usepackage[numbers,sort&compress]{natbib}

%%%%%%%%%%%%%
% Highlight
%%%%%%%%%%%%%
\usepackage{soul}
\soulregister\cite7
\soulregister\autoref7
\soulregister\code7
\soulregister\toolname7
\soulregister\ref7
\soulregister\pageref7
\usepackage{hhline}
\definecolor{lightyellow}{RGB}{250, 250, 180}
\definecolor{HLYELLOW}{RGB}{240, 127, 0}
\definecolor{hlyellow}{RGB}{240, 127, 0}
\sethlcolor{lightyellow}

%%%%%%%%%%%%%
% Algorithm stuff
%%%%%%%%%%%%%
\usepackage{algorithm}
\usepackage[noend]{algpseudocode}
\usepackage{pifont}
\usepackage{amsmath}
% clash with acmart
\let\Bbbk\relax
\renewcommand{\algorithmicforall}{\textbf{for each}}
% Various for each, for each in parallel, for each mod P...
\def\ForEach{\ForAll}
\algdef{SE}[FOREACHIN]{ForEachIn}{EndForEachIn}[2]{\algorithmicfor\ \textbf{each}\ #1\ \textbf{in}\ #2\ \textbf{do}}{\algorithmicend\ \algorithmicfor}%
%\algdef{SE}[FOREACHIN]{ForEachIn}{EndForEachIn}[2]{\algorithmicfor\ \textbf{each}\ #1\ \textbf{in}\ #2\ \textbf{do}}{}%
\algdef{SE}[FOREACHP]{ForEachP}{EndForEachP}[1]{\algorithmicfor\ \textbf{each}\ #1\ \textbf{in parallel do}}{\algorithmicend\ \algorithmicfor}%
%\algdef{SE}[FOREACHP]{ForEachP}{EndForEachP}[1]{\algorithmicfor\ \textbf{each}\ #1\ \textbf{in parallel do}}{}%
\algdef{SE}[FOREACHINP]{ForEachInP}{EndForEachInP}[2]{\algorithmicfor\ \textbf{each}\ #1\ \textbf{in}\ #2\ \textbf{in parallel do}}{\algorithmicend\ \algorithmicfor}%
\algdef{SE}[FORMOD]{ForMod}{EndForMod}[3]{\algorithmicfor\ #1\ \textbf{from}\ #2\ \textbf{to}\ #3\ \textbf{do}}{\algorithmicend\ \algorithmicfor}%
\algdef{SE}[FORMOD]{ForModS}{EndForModS}[4]{\algorithmicfor\ #1\ \textbf{from}\ #2\ \textbf{to}\ #3\ \textbf{step}\ #4\ \textbf{do}}{\algorithmicend\ \algorithmicfor}%
\algdef{SE}[FORMODP]{ForModPS}{EndForModPS}[3]{\algorithmicfor\ #1\ \textbf{from}\ #2\ \textbf{to}\ #3\ \textbf{in parallel do}}{\algorithmicend\ \algorithmicfor}%
\algdef{SE}[FORMODP]{ForModP}{EndForModP}[4]{\algorithmicfor\ #1\ \textbf{from}\ #2\ \textbf{to}\ #3\ \textbf{step} #4\ \textbf{in parallel do}}{\algorithmicend\ \algorithmicfor}%

% avoid displaying empty lines when there's nothing at the end
\algtext*{EndForEachIn}
\algtext*{EndForEachP}

\algdef{SE}[VARIABLES]{Variables}{EndVariables}{\algorithmicvariables}
  {\algorithmicend\ \algorithmicvariables}
\algnewcommand{\algorithmicvariables}{\textbf{global}}
\algnewcommand{\LineComment}[1]{\State \(\triangleright\) #1}
% https://tex.stackexchange.com/questions/50908/algorithms-and-boolean-operator-casuing-undefined-control-sequence-error
\algnewcommand{\And}{\textbf{and}\xspace}


%%%%%%%%%%%%%
% Tikz nodes in the text
%%%%%%%%%%%%%
% https://tex.stackexchange.com/questions/7032/good-way-to-make-textcircled-numbers
\usepackage{tikz}
\usepackage{pifont}
\usetikzlibrary{shapes}
\DeclareRobustCommand*\circled[1]{\tikz[baseline=(char.base)]{
\node[shape=circle,fill=brown,draw=brown,inner sep=0pt] (char) {\textcolor{white}{\small\textbf{#1}}};}}
\DeclareRobustCommand*\circledColor[2]{\tikz[baseline=(char.base)]{
    \node[shape=circle,fill=#2,draw=#2,inner sep=1pt] (char) {\textcolor{white}{\small\textbf{#1}}};}}
\DeclareRobustCommand*\circledColorSmall[2]{\tikz[baseline=(char.base)]{
    \node[shape=circle,fill=#2,draw=#2,inner sep=0pt] (char) {\textcolor{white}{\footnotesize\textbf{#1}}};}}
\DeclareRobustCommand*\nodeColor[3]{%
  \tikz[baseline=(char.base)]{%
    \node[shape=#3,fill=#2,draw=#2,inner sep=0pt] (char) {\textcolor{white}{\small\textbf{#1}}};%
  }%
}
\DeclareRobustCommand*\nodeColorSmall[3]{%
  \tikz[baseline=(char.base)]{%
    \node[shape=#3,fill=#2,draw=#2,inner sep=0pt] (char) {\textcolor{white}{\footnotesize\textbf{#1}}};%
  }%
}

\usepackage{float}
%%%%%%%%%%%%%%
% C++ listings
%%%%%%%%%%%%%%
\usepackage{listings}
\newfloat{codeblock}{H}{myc}
%\newcommand{\code}[1]{\figure{#1}}
\definecolor{darkblue}{rgb}{0,0,.6}
\definecolor{darkred}{rgb}{.6,0,0}
%\definecolor{darkgreen}{rgb}{0,.6,0}
\definecolor{darkgreen}{rgb}{0,.5,0}
\definecolor{red}{rgb}{.98,0,0}
\definecolor{gray}{rgb}{.6,.6,.6}
\lstloadlanguages{C++}
% Settings for the lstlistings environment
\lstset{
	language=C++,                       % choose the language of the code
  basicstyle=\footnotesize\ttfamily\linespread{0.8},  % the size of the fonts that are used for the
	%basicstyle=\scriptsize,  % the size of the fonts that are used for the
	% code
	numbers=left,                       % where to put the line-numbers
	numberstyle=\tiny,                  % the size of the fonts that are used for the
	% line-numbers
	stepnumber=1,                       % the step between two line-numbers. If it's
	% 1 each line will be numbered
	numbersep=5pt,                      % how far the line-numbers are from the code
	%backgroundcolor=\color{gray},      % choose the background color. You must add
	% \usepackage{color}
	showspaces=false,                   % show spaces adding particular underscores
	showstringspaces=false,             % underline spaces within strings
	showtabs=false,                     % show tabs within strings adding particular
	% underscores
	keywordstyle=\bfseries\color{black},  % color of the keywords
	commentstyle=\color{darkgreen},     % color of the comments
	stringstyle=\color{darkred},        % color of strings
	captionpos=b,                       % sets the caption-position to top
	tabsize=2,                          % sets default tabsize to 2 spaces
	frame=tb,                       % adds a frame around the code
	breaklines=true,                    % sets automatic line breaking
	breakatwhitespace=false,            % sets if automatic breaks should only happen
	% at whitespace
	%escapechar=\%,                      % toggles between regular LaTeX and listing
	%belowskip=0.3cm,                    % vspace after listing
	morecomment=[s][\bfseries]{struct}{\ },
	morecomment=[s][\bfseries]{class}{\ },
	morecomment=[s][\bfseries]{public:}{\ },
	morecomment=[s][\bfseries]{public}{\ },
	morecomment=[s][\bfseries]{protected:}{\ },
	morecomment=[s][\bfseries]{private:}{\ },
	morecomment=[s][\bfseries\color{black}]{operator+}{\ },
	xleftmargin=0.1cm,
  literate={\%}{\%}{1},
	%xrightmargin=0.1cm,
  aboveskip=0pt,
  belowskip=0pt,
}
