

\newif\iftr     % Full arXiv technical report
\newif\ifconf   % Size-constrainted Submission to a conf or journal
\newif\ifnonb   % Non blind submission

%%%%%%%%%%%%%
% Various
%%%%%%%%%%%%%
\usepackage[utf8]{inputenc}
\newcommand{\code}[1]{\texttt{#1}}
% binary units
\usepackage[binary-units=true]{siunitx}
\usepackage{xspace}

%%%%%%%%%%%%%
% Notes
%%%%%%%%%%%%%
\usepackage{collab}
\collabAuthor{copik}{blue}{Marcin}
\collabAuthor{htor}{red}{Torsten}

%%%%%%%%%%%%%
% Highlight
%%%%%%%%%%%%%
\usepackage{soul}
\usepackage{xcolor}
\soulregister\cite7
\soulregister\autoref7
\soulregister\code7
\soulregister\toolname7
\soulregister\ref7
\soulregister\pageref7
\usepackage{hhline}
\definecolor{lightyellow}{RGB}{250, 250, 180}
\definecolor{HLYELLOW}{RGB}{240, 127, 0}
\definecolor{hlyellow}{RGB}{240, 127, 0}
\sethlcolor{lightyellow}

%%%%%%%%%%%%%
% Algorithm stuff
%%%%%%%%%%%%%
\usepackage{algorithm}
\usepackage{algpseudocode}
\usepackage{pifont}
\usepackage{amsmath}
% clash with acmart
\let\Bbbk\relax
\renewcommand{\algorithmicforall}{\textbf{for each}}
% Various for each, for each in parallel, for each mod P...
\def\ForEach{\ForAll}
\algdef{SE}[FOREACHIN]{ForEachIn}{EndForEachIn}[2]{\algorithmicfor\ \textbf{each}\ #1\ \textbf{in}\ #2\ \textbf{do}}{\algorithmicend\ \algorithmicfor}%
\algdef{SE}[FOREACHP]{ForEachP}{EndForEachP}[1]{\algorithmicfor\ \textbf{each}\ #1\ \textbf{in parallel do}}{\algorithmicend\ \algorithmicfor}%
\algdef{SE}[FOREACHINP]{ForEachInP}{EndForEachInP}[2]{\algorithmicfor\ \textbf{each}\ #1\ \textbf{in}\ #2\ \textbf{in parallel do}}{\algorithmicend\ \algorithmicfor}%
\algdef{SE}[FORMOD]{ForMod}{EndForMod}[3]{\algorithmicfor\ #1\ \textbf{from}\ #2\ \textbf{to}\ #3\ \textbf{do}}{\algorithmicend\ \algorithmicfor}%
\algdef{SE}[FORMOD]{ForModS}{EndForModS}[4]{\algorithmicfor\ #1\ \textbf{from}\ #2\ \textbf{to}\ #3\ \textbf{step}\ #4\ \textbf{do}}{\algorithmicend\ \algorithmicfor}%
\algdef{SE}[FORMODP]{ForModPS}{EndForModPS}[3]{\algorithmicfor\ #1\ \textbf{from}\ #2\ \textbf{to}\ #3\ \textbf{in parallel do}}{\algorithmicend\ \algorithmicfor}%
\algdef{SE}[FORMODP]{ForModP}{EndForModP}[4]{\algorithmicfor\ #1\ \textbf{from}\ #2\ \textbf{to}\ #3\ \textbf{step} #4\ \textbf{in parallel do}}{\algorithmicend\ \algorithmicfor}%

\algdef{SE}[VARIABLES]{Variables}{EndVariables}{\algorithmicvariables}
  {\algorithmicend\ \algorithmicvariables}
  \algnewcommand{\algorithmicvariables}{\textbf{global}}
\algnewcommand{\LineComment}[1]{\State \(\triangleright\) #1}
% https://tex.stackexchange.com/questions/50908/algorithms-and-boolean-operator-casuing-undefined-control-sequence-error
\algnewcommand{\And}{\textbf{and}\xspace}

%%%%%%%%%%%%%
% Tikz nodes in text
%%%%%%%%%%%%%
% https://tex.stackexchange.com/questions/7032/good-way-to-make-textcircled-numbers
\usepackage{tikz}
\DeclareRobustCommand*\circled[1]{\tikz[baseline=(char.base)]{
\node[shape=circle,fill=brown,draw=brown,inner sep=0pt] (char) {\textcolor{white}{\small\textbf{#1}}};}}
\DeclareRobustCommand*\circledColor[2]{\tikz[baseline=(char.base)]{
    \node[shape=circle,fill=#2,draw=#2,inner sep=1pt] (char) {\textcolor{white}{\small\textbf{#1}}};}}


